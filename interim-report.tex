%text.tex

\documentclass{article}
\usepackage{times}
\usepackage{graphicx}
\usepackage{amsmath}
\usepackage{url}
\usepackage{amssymb}
\usepackage[top =2in]{geometry}
\DeclareGraphicsExtensions{.pdf}
\title{CSE881 - Project Intermediate Report}
\author{Sorrachai Yingchareonthawornchai, Vaibhav Sharma \\
 \texttt{yingchar@msu.edu}, \texttt{sharmav6@msu.edu}\\
}
\begin{document}
\maketitle

\section*{ Fraudulent Resume Detection }
\textbf{Project Type:} Prototype Development

\section{Revised Abstract} Given a resume, return a value with links to original resume(s) or evidences of plagiarism from resumes in database. \\
We define a resume as a set of sections. A section contains a collection of sentences. We assume that the document has text format. We regard techniques in natural language processing as not in scope of this project. The problem statement of this project is also requested as a solution by a company which will be providing us with the data needed for this project.

\section{Preliminary work}
We have done the following preliminary works.
\begin{enumerate}
\item Preprocessing: partition resume into a set of sections. Each
  section is represented as a list of bags of keywords (common words eliminiated).
\item Feature Extraction Design: we have planned to use 5 features
  (indicators). One technical feature is section-level plagiarism checking. We recast
  plagiarism checking as a bipartite graph maximum matching problem.  A
  set of vertices in first bi-partition corresponds to a list of word
  lines in the first section. The second set of vertices is second
  bi-partition, similar to
  the first one. Edge is weighted by Jaccard similarity between two
  word lines. The extracted feature will be discretized into degree of
  confidence. 
\item Classification Method: we planned to use k-NN classifier as a baseline
  to compare against k-NN combined with Naive Bayes' classifier.
\end{enumerate}

\section{Future Work}

\subsection{Tasks}
\subsubsection{Data Mining Tasks}
\begin{itemize}
	\item baseline: search for k-nearest pairs using jaccard similarity or q-gram. 
	\item our approach: (int t)
	\begin{enumerate}
		\item search for \textbf{t}-nearest pairs using same as above
		\item among \textbf{t} candidates, apply Naive Bayes' classifier using indicators
	\end{enumerate}
	\item indicators:
	\begin{enumerate}
		\item If resumes belongs to different owners, plagiarism between sections (use graph maximum matching, parameter: threshold)
		\item If resumes belongs to different owners, match university name with list of faked universities (parameter: list of faked university) 
		\item If resumes belongs to different owners, match DOB vs. claimed experience
		\item If resumes belongs to same owner, inverse of
                  plagiarism between sections (parameter: threshold)
	\end{enumerate}
\end{itemize}
 \subsubsection{Website Development}
The website has two pages: upload\_resume.php and display\_result.php. \\
 After classification, we also show indicators with confidence level
 (if any). For example, if there is plagiarism between two sections,
 the website also show links to sections where match occurs.
\subsection{Timeline}
\begin{itemize}
\item Classifier Design \& Preprocessing - completed - Oct 24
\item Baseline, indicator functions - Nov 11
\item Train Naive Bayes model - Nov 18
\item Prototype website design \& development - Nov 27
\item Bug fixing \& Final report - Dec 2
\end{itemize}

\section{Contribution}
\begin{itemize}
\item Sorrachai Yingchareonthawornchai - he has designed section
  level-plagiarism checker, and reviewed duplicate detection
  literatures. He will complete implementation of feature extractions
  namely bipartite graph maximum matching, and prototype development.
\item Vaibhav Sharma - was responsible for the preprocessing task, will be responsible for establishing the baseline using Jaccard Similary and kNN metrics, clubbing all base indicator functions together and passing it to our Naive Bayes classifier, will be partially responsible for designing and implementing the prototype
\end{itemize}

\end{document}
