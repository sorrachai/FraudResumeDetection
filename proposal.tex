%text.tex

\documentclass{article}
\usepackage{times}
\usepackage{graphicx}
\usepackage{amsmath}
\usepackage{url}
\usepackage{amssymb}
\DeclareGraphicsExtensions{.pdf}
\title{CSE881 - Project Proposal}
\author{Sorrachai Yingchareonthawornchai, Vaibhav Sharma \\
 \texttt{yingchar@msu.edu}, \texttt{sharmav6@msu.edu}\\
}
\begin{document}
\maketitle
\section{Fraudulent Resume Detection}
Project type: Prototype Development

(Importance of the project as a practical applications) ...
\\
Problem statement: given a resume, determine whether or not the resume
is fabricated. \\
Goal: returns a value for a given resume with links or evidences to
original resume in database. \\

We define a resume as a set of sections. A section contains a collection of
sentences. We assume that the document has XML format. We regard
techniques in natural language processing as not in scope of this
project. 

\textbf{Overview of tasks.}\\
In the preprocessing step, we create a document parser that takes in
XML documents as input. The output is a set of sections containing
topics and its sentences.  The next step is to identify indicators
that we need to perform detection. Indicators have varieties of
challenges and intelligence. For example, one simple indicator would be matching
university name with our fake university list. Another example would
be check degree of plagiarism. Next example, checking consistency
regarding description of the company against company name (e.g.,
``Google, description: a nuclear energy plants'' does not make
sense). Each indicator could be a data mining task. Once we have
all indicators, we regards them as feature of the document. We then
perform classification to output a degree of fraudulence and
confidence level. Finally, we will build an interactive system that
helps visualize the model and outputs. 

Brief summary of tasks: 
\begin{enumerate}
\item Document parser (Preprocessing)
\item Indication function (Feature extraction)
\item Classification using extracted features
\item An Interactive system that users can interact with
\end{enumerate}
\end{document}
